%% beamer Präsentation
%%====================
%% Informationen:
%%---------------
% % % % Befehle und Links
    %%Datei verkleinern
        %gs -dBATCH -dNOPAUSE -sDEVICE=pdfwrite -dTextAlphaBits=4 -dGraphicsAlphaBits=4 -r75 -sOutputFile=/home/fon/daten/praes/tex/nakoinz_berlin_dak8_2014_v02k.pdf   /home/fon/daten/praes/tex/nakoinz_berlin_dak8_2014_v02.pdfwrite
    %% Frabnamen: http://www.namsu.de/Extra/pakete/Xcolor.html  

% % % % Präsentationsinfo:
    %
    %
    %

%% Vorspann
%%----------
%\documentclass[12pt, t, dvipsnames, aspectratio=1610]{beamer}   % t bewirkt Ausrichtung oben statt mittig (ggf. mit klammern löschen)

%\documentclass[notes, 10pt, t, dvipsnames, aspectratio=1610]{beamerarticle}   % t bewirkt Ausrichtung oben statt mittig (ggf. mit klammern löschen)
%\documentclass[10pt, t, handout]{beamer} 
\usepackage[ngerman,english]{babel}
%\usepackage[utf8]{inputenc}
\usepackage[T1]{fontenc}
\usepackage{amsmath,amsfonts,amssymb}
%\usepackage[usenames]{color}
\usepackage{color}
\usepackage{hyperref}
\usepackage{wasysym}
\usepackage{graphicx}
\usepackage[normalem]{ulem}
\usepackage{enumerate}
%\usepackage{enumitem,multicol}   % dann gehen normale listen nicht
\usepackage{microtype}
\usepackage{setspace}

%\usepackage{DejaVuSansCondensed}
%\renewcommand*\familydefault{\sfdefault}

 % nur bei xetex
%	\usepackage{fontspec} 
    %\usepackage[cm-default]{fontspec}
 %   \usepackage{xunicode}
 %   \usepackage{xltxtra}
	%\setmainfont{DejaVu Sans Condensed}
%    \setmainfont{FreeSerif}
%    \setsansfont{FreeSans}
%    \setmonofont{FreeMono}
    
% % % % % Definitionen
%     \usepackage{thmbox}
%     %\newtheorem[L]{thmS}{Definition}  
%     \newtheorem{Def}{Definition}
    
% % % % % Graphik    
    %\usepackage{fp}
    \usepackage{tikz}
    %\usepackage{xcolor}
    %% TikZ-Bibliotheken
    %\usetikzlibrary{arrows}
    %\usetikzlibrary{shapes}
    
% % % % % Inhaltsverzeichnis Helligkeit
     \setbeamertemplate{section in toc shaded}[default][80]

% % % % % Rahmen
     \setlength{\fboxsep}{0pt}  % Rahmenabstand für Abbildungen

% % % % % Code Listings
%     % % listings
%         \usepackage{listings} % https://en.wikibooks.org/wiki/LaTeX/Source_Code_Listings
%         \lstset{ %
%             basicstyle=\footnotesize,       % the size of the fonts that are used for the code
%             showspaces=false,               % show spaces adding particular underscores
%             showstringspaces=false,         % underline spaces within strings
%             showtabs=false,                 % show tabs within strings adding particular underscores
%             frame=single,                   % adds a frame around the code
%             tabsize=4,                      % sets default tabsize to 2 spaces
%             breaklines=true,                % sets automatic line breaking
%             breakatwhitespace=false,        % sets if automatic breaks should only happen at whitespace
%         }

%     % % minted
%         \usepackage{minted}      % für CodeFragmente
%         \usemintedstyle{tango}
%         \renewcommand*\familydefault{\sfdefault}

% % % % % Block
    \setbeamertemplate{blocks}[rounded][shadow=false] 
    %\setbeamertemplate{blocks}[rounded][shadow=true] 

% % % % %  Item
    \setbeamertemplate{items}[circle] 
    \setbeamertemplate{items}[triangles] 

% % % % % Splaten
     \setlength{\columnsep}{0.5cm}       % Abstand zwischen den Spalten

% % % % % Farben
%     %% Faben allgemein
    %\definecolor{papier}{RGB}{237,227,117}
    %\definecolor{papier}{RGB}{231,211,177}
    \definecolor{papier}{RGB}{236,224,176}
    \definecolor{krot}{RGB}{229,20,60}
    \definecolor{hellblau}{RGB}{110,130,160}
    \definecolor{dunkelblau}{RGB}{45,60,90}
    \definecolor{grau100}{RGB}{100,100,100}
    \definecolor{grau120}{RGB}{120,120,120}
    \definecolor{grau130}{RGB}{140,140,140}
    \definecolor{grau140}{RGB}{140,140,140}
    \definecolor{grau150}{RGB}{140,140,140}
    \definecolor{grau160}{RGB}{160,160,160}
    \definecolor{grau170}{RGB}{170,170,170}
    \definecolor{grau180}{RGB}{180,180,180}
    \definecolor{grau200}{RGB}{200,200,200}
    
%     % Standardfarbe:
    \definecolor{beamer@blendedblue}{RGB}{45,60,90}
    \usecolortheme[RGB={45,60,90}]{structure} 
    \setbeamercolor{frametitle}{bg=}
    \setbeamercolor{title}{bg=}
    
%     %% äußere Farben:
    \mode<presentation>
        \setbeamercolor*{palette primary}{use=structure,fg=black,bg=grau150} % Titel und unten links
        \setbeamercolor*{palette secondary}{use=structure,fg=black,bg=grau140} % oben rechts und unten rechts
        \setbeamercolor*{palette tertiary}{use=structure,fg=black,bg=grau120}  % oben links
        \setbeamercolor*{palette quaternary}{fg=white,bg=black}
        \setbeamercolor*{palette sidebar secondary}{fg=white}
        \setbeamercolor*{palette sidebar quaternary}{fg=white}
        %\setbeamercolor*{titlelike}{parent=palette primary}
        \setbeamercolor*{separation line}{}
        \setbeamercolor*{fine separation line}{}
        \setbeamercolor{item projected}{use=item,fg=dunkelblau,bg=item.fg!dunkelblau}
    \mode
    <all>
    
    %% Blockfarbe
    \mode<presentation>
        \setbeamercolor{block title}{use=structure,fg=white!30!hellblau,bg=black!50!hellblau}
        \setbeamercolor{block title alerted}{use=red,fg=white,bg=alerted text.grau100!30!red}
        \setbeamercolor{block title example}{use=grau100,fg=white,bg=grau100}
        \setbeamercolor{block body}{use=papier,bg=grau100!10!papier}
        \setbeamercolor{block body alerted}{use=papier,bg=grau100!10!papier}
        \setbeamercolor{block body example}{use=papier,bg=grau100!10!papier}
    \mode
    <all>

% % % % % Leisten und Schalter
%     \beamertemplatenavigationsymbolsempty  % navigation ausschalten
%     \setbeamertemplate{headline}[infolines]
%     \setbeamertemplate{footline}[frame number]
%     %\setbeamerfont{headline}{family=\ttfamily} 
%     %\setbeamerfont{footline}{family=\ttfamily} 
%     %\setbeamerfont{page number in head/foot}{family=\ttfamily} 
%     \setbeamerfont{headline}{family=\sffamily} 
%     \setbeamerfont{footline}{family=\sffamily} 
%     \setbeamerfont{page number in head/foot}{family=\sffamily} 
%     %\fontfamily{\familydefault}
%     %\fontseries{\seriesdefault}

%% Kopfleiste
     \defbeamertemplate*{headline}{infolines theme}
     { \leavevmode%
       \hbox{%
         \begin{beamercolorbox}[wd=.7\paperwidth,ht=2.25ex,dp=1ex,left]{section in head/foot}%
           \usebeamerfont{section in head/foot}%\insertsectionhead
           \hspace*{2ex}
           \insertshorttitle
           % \inserttitle
         \end{beamercolorbox}%
         \begin{beamercolorbox}[wd=.3\paperwidth,ht=2.25ex,dp=1ex,right]{subsection in head/foot}%
           \usebeamerfont{subsection in head/foot}\hspace*{2ex}%\insertsubsectionhead
           \insertshortauthor
           \hspace*{2ex}
         \end{beamercolorbox}}%
       \vskip0pt%
     }

    %% Fußleiste
    % \defbeamertemplate*{footline}{infolines theme}
    % { \leavevmode%
    %  \hbox{%
    %  \begin{beamercolorbox}[wd=.8\paperwidth,ht=2.25ex,dp=1ex,center]{author in head/foot}%
    %    % hier kommt der Ablauf hintergrund %%%%%%%%%%%%%%%%%%%%%%%%%%%%%%%%%%%%%%%%%%%%%%%
    %    A - B - C - D
    %    %%%%%%%%%%%%%%%%%%%%%%%%%%%%%%%%%%%%%%%%%%%%%%%%%%%%%%%%%%%%%%%%%%%%%%%%%%%%%%%%%%%
    %  \end{beamercolorbox}%
    %  \begin{beamercolorbox}[wd=.2\paperwidth,ht=2.25ex,dp=1ex,right]{date in head/foot}%
    %    %\usebeamerfont{date in head/foot}\insertshortdate{}\hspace*{2em}
    %    \insertframenumber{} / \inserttotalframenumber\hspace*{2ex} 
    %  \end{beamercolorbox}}%
    %  \vskip0pt%
    % }

    \defbeamertemplate*{footline}{infolines theme}
    { \leavevmode%
      \hbox{%
      \begin{beamercolorbox}[wd=.8\paperwidth,ht=2.25ex,dp=1ex,center]{author in head/foot}%
        \insertsectionhead
      \end{beamercolorbox}%
      \begin{beamercolorbox}[wd=.2\paperwidth,ht=2.25ex,dp=1ex,right]{date in head/foot}%
        %\usebeamerfont{date in head/foot}\insertshortdate{}\hspace*{2em}
        \insertframenumber{} / \inserttotalframenumber\hspace*{2ex} 
      \end{beamercolorbox}}%
      \vskip0pt%
    }  
    
    % % % % %  Fortschrittsbalken ueber der Fusszeile
    \definecolor{pbarcol}{RGB}{45,60,90}  %{0.7 0.7 0.7}
    \makeatletter
    \addtobeamertemplate{footline}{%
      \color{pbarcol}         % to color the progressbar
      \hspace*{-\beamer@leftmargin}%
      \rule{\beamer@leftmargin}{2pt}%
      \rlap{\rule{\dimexpr
          \beamer@startpageofframe\dimexpr
          \beamer@rightmargin+\textwidth\relax/\beamer@endpageofdocument}{1pt}}
      % next 'empty' line is mandatory!
    
      \vspace{0\baselineskip}
      {}
    }

% % % % Rand:
    \setbeamersize{text margin left=5mm, text margin right=5mm} 
    %\geometry{   paperwidth= vmargin=0pt,   head=5pt, %   headsep=0pt,%   foot=0.5cm % 
    %    }
    %\setbeamertemplate{frametitle}{
    %    \vspace{-2cm}\\
    %    \insertframetitle
    %}

\subtitle{HS/Ü: xxxxxxxxx\\
  \vspace{1em}
  Institut für Ur- und Frühgeschichte\\
  Christian-Albrechts-Universität zu Kiel\\
  \vspace{1em}
  Oliver Nakoinz | Daniel Knitter\\
  \vspace{1em}
  Sommersemester 2017
}

                                     %%%%%%
                              %%%%%%%%%%%%%%%%%%%%
               %%%%%%%%%%%%%%%%%%%%%%%%%%%%%%%%%%%%%%%%%%%%%%%%%%
%%%%%%%%%%%%%%%%%%%%%%%%%%%%%%%%%%%%%%%%%%%%%%%%%%%%%%%%%%%%%%%%%%%%%%%%%%%%%%%%

